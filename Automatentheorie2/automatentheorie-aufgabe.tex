
\documentclass{article}
%encoding
%--------------------------------------
\usepackage[utf8]{inputenc}
\usepackage[T1]{fontenc}
%--------------------------------------
 
%German-specific commands
%--------------------------------------
\usepackage[ngerman]{babel}
\usepackage{csquotes}
%--------------------------------------

%Margins
%--------------------------------------
\usepackage{geometry}
 \geometry{
 a4paper,
 total={170mm,257mm},
 left=20mm,
 top=20mm,
 }
 
%Pictures
%--------------------------------------
\usepackage{graphicx}
\graphicspath{ {./Pictures/} }
\usepackage{tikz}
\usepackage{subcaption}
\usepackage{float}
\usepackage{wrapfig}
%--------------------------------------

%finite automata
%--------------------------------------
\usetikzlibrary{automata,positioning,arrows}
\tikzset{->,  % makes the edges directed
>=stealth', % makes the arrow heads bold
node distance=3cm, % specifies the minimum distance between two nodes. Change if necessary.
every state/.style={thick, fill=gray!10}, % sets the properties for each ’state’ node
initial text=$ $, % sets the text that appears on the start arrow
}

%tables
%--------------------------------------
\usepackage{array}

%math
%--------------------------------------
\usepackage[fleqn]{amsmath}
\usepackage{amssymb}
\usepackage{amsfonts}
%--------------------------------------

%Frames
%--------------------------------------
\usepackage{framed}

%Own math commands
%--------------------------------------
\newcommand{\abs}[1]{\lvert#1\rvert}

%Colors
%--------------------------------------
\usepackage{xcolor}
\definecolor{blue-violet}{rgb}{0.54, 0.17, 0.89}
\definecolor{codegreen}{rgb}{0,0.6,0}
\definecolor{codegray}{rgb}{0.5,0.5,0.5}
\definecolor{codepurple}{rgb}{0.58,0,0.82}
\definecolor{backcolour}{rgb}{0.95,0.95,0.92}

%--------------------------------------
%\usepackage{multicol}
\usepackage{paracol}
\usepackage[shortlabels]{enumitem}

%Aufgaben
%--------------------------------------
\usepackage{amsthm}
\newtheorem{aufgabe}{Aufgabe}[section]
\newtheorem{definition}{Definition}[section]
\newtheorem{beispiel}{Beispiel}[section]
%--------------------------------------

%Listings
%--------------------------------------
\usepackage{ulem}
\usepackage{listings}
 
\lstdefinestyle{mystyle}{
    backgroundcolor=\color{backcolour},   
    commentstyle=\color{codegreen},
    keywordstyle=\color{magenta},
    numberstyle=\tiny\color{codegray},
    stringstyle=\color{codepurple},
    basicstyle=\ttfamily\footnotesize,
    breakatwhitespace=false,         
    breaklines=true,                 
    captionpos=b,                    
    keepspaces=true,                 
    numbers=left,                    
    numbersep=5pt,                  
    showspaces=false,                
    showstringspaces=false,
    showtabs=false,                  
    tabsize=2,
}
 
\lstset{style=mystyle,moredelim=[is][\sout]{|}{|}}
%--------------------------------------



\title{Automatentheorie -- eine Aufgabe}
\author{Alexandra Maximova}
\date{16. November 2020}

\begin{document}

\maketitle


\section*{Aufgabe}

Entwirf einen endlichen Automaten, der die Sprache \(L = \{ x1100y \mid x, y \in \{0,1\}^*\}\) erkennt, und bestimme die Bedeutung aller Zustandsklassen.

\section*{Lösung}
\begin{figure}[H]
\centering
\begin{tikzpicture}

  \node[state,initial] (q_{lambda}) {$q_{\lambda}$};
  \node[state] (q_1) [right of=q_{lambda}] {$q_1$};
  \node[state] (q_{11}) [right of=q_1] {$q_{11}$};
  \node[state] (q_{110}) [right of=q_{11}] {$q_{110}$};
  \node[state,accepting] (q_{1100}) [right of=q_{110}] {$q_{1100}$};

  \path (q_{lambda}) edge[bend left, above]  node {$1$}   (q_1)
        (q_{lambda}) edge[loop above]        node {$0$}   (q_{lambda})
        (q_1)        edge[bend left, below]  node {$0$}   (q_{lambda})
        (q_1)        edge[bend left, above]  node {$1$}   (q_{11})
        (q_{11})     edge[bend left, above]  node {$0$}   (q_{110})
        (q_{11})     edge[loop above]        node {$1$}   (q_{11})
        (q_{110})    edge[bend left, above]  node {$0$}   (q_{1100})
        (q_{110})    edge[bend left, below]  node {$1$}   (q_1)
        (q_{1100})   edge[loop above]        node {$0,1$} (q_{1100});
\end{tikzpicture}
\end{figure}

\begin{alignat*}{2}
&\text{Klasse}[q_{\lambda}] && = \{\lambda\} \cup \{x0 \mid x \in \{0,1\}^* \text{ und $x0$ enthält das Teilwort $1100$ nicht}\} \\
&\text{Klasse}[q_{1}] && = \{x1 \mid x \in \{0,1\}^* \text{ und $x1$ enthält das Teilwort $1100$ nicht}\}  \\
&\text{Klasse}[q_{11}] && = \{x11 \mid x \in \{0,1\}^* \text{ und $x11$ enthält das Teilwort $1100$ nicht}\} \\
&\text{Klasse}[q_{110}] && = \{x110 \mid x \in \{0,1\}^* \text{ und $x110$ enthält das Teilwort $1100$ nicht}\} \\
&\text{Klasse}[q_{1100}] && = \{x1100y \mid x,y \in \{0,1\}^*\} = L \\
\end{alignat*}

\section*{Hinweise für die Bewertung}
\begin{tabular}{ |p{0.8\textwidth}|r| } 
 \hline
Der Anfangszustand ist markiert & \(\ldots / 0.5\) \\ 
 \hline
Die akzeptierende Zustände sind markiert & \(\ldots / 0.5\) \\ 
 \hline
Der Automat ist vollständig (d.h. aus jedem Zustand gibt es einen Zustandsübergang für jedes Symbol aus dem Alphabet) & \(\ldots / 1.0\) \\ 
 \hline
Der Automat akzeptiert das Wort \(1100\) (und hat mindestens fünf Zustände) & \(\ldots / 0.5\) \\ 
 \hline
Der Automat akzeptiert alle Wörter, die \(1100\) als Präfix haben, (und hat mindestens fünf Zustände) & \(\ldots / 1.0\) \\ 
 \hline
Der Automat akzeptiert die Wörter \(11100\) und \(1101100\), (und hat mindestens fünf Zustände) & \(\ldots / 1.0\) \\ 
 \hline
Der Automat akzeptiert alle Wörter, die \(1100\) als Teilwort enthalten (und hat mindestens fünf Zustände) & \(\ldots / 1.0\) \\ 
 \hline
Der Automat akzeptiert das Wort \(110100\) nicht & \(\ldots / 1.0\) \\ 
 \hline
Der Automat akzeptiert keine Wörter, die nicht in L sind & \(\ldots / 1.0\) \\ 
 \hline
Die Beschreibungen der Zustandsklassen sind korrekt für den gezeichneten Automaten oder für den Automaten aus der Musterlösung.  & \(\ldots / 2.5\) \\ 
 \hline
 \hline
\textbf{Total} & \(\dots /10.0\) \\
\hline
\end{tabular}

\subsection*{Beispiele}

\subsubsection*{Automat, der alles akzeptiert}
\begin{figure}[H]
\centering
\begin{tikzpicture}

  \node[state,initial,accepting] (q_0) {$q_{0}$};
 
  \path (q_0) edge[loop above]  node {$0,1$}   (q_0);
\end{tikzpicture}
\end{figure}
\begin{flalign*}
\text{Klasse}[q_0] &= \{0,1\}^*
\end{flalign*}

\textcolor{red}{Nach dem Bewertungsschema ist diese Lösung \textbf{2.5} Punkte wert.}

\subsubsection*{Automat, der 11100 nicht akzeptiert}
\begin{figure}[H]
\centering
\begin{tikzpicture}

  \node[state,initial] (q_{lambda}) {$q_{\lambda}$};
  \node[state] (q_1) [right of=q_{lambda}] {$q_1$};
  \node[state] (q_{11}) [right of=q_1] {$q_{11}$};
  \node[state] (q_{110}) [right of=q_{11}] {$q_{110}$};
  \node[state,accepting] (q_{1100}) [right of=q_{110}] {$q_{1100}$};

  \path (q_{lambda}) edge[bend left, above]  node {$1$}   (q_1)
        (q_{lambda}) edge[loop above]        node {$0$}   (q_{lambda})
        (q_1)        edge[bend left, below]  node {$0$}   (q_{lambda})
        (q_1)        edge[bend left, above]  node {$1$}   (q_{11})
        (q_{11})     edge[bend left, above]  node {$0$}   (q_{110})
        (q_{11})     edge[bend left, below, draw=red]        node {$\color{red}{1}$}   (q_1)
        (q_{110})    edge[bend left, above]  node {$0$}   (q_{1100})
        (q_{110})    edge[bend left, below]  node {$1$}   (q_1)
        (q_{1100})   edge[loop above]        node {$0,1$} (q_{1100});
\end{tikzpicture}
\end{figure}

\begin{alignat*}{2}
&\text{Klasse}[q_{\lambda}] && = \{\lambda\} \cup \{x0 \mid x \in \{0,1\}^* \text{ und $x0$ enthält das Teilwort $1100$ nicht}\} \\
&\text{Klasse}[q_{1}] && = \{x1 \mid x \in \{0,1\}^* \text{ und $x1$ enthält das Teilwort $1100$ nicht}\}  \\
&\text{Klasse}[q_{11}] && = \{x11 \mid x \in \{0,1\}^* \text{ und $x11$ enthält das Teilwort $1100$ nicht}\} \\
&\text{Klasse}[q_{110}] && = \{x110 \mid x \in \{0,1\}^* \text{ und $x110$ enthält das Teilwort $1100$ nicht}\} \\
&\text{Klasse}[q_{1100}] && = \{x1100y \mid x,y \in \{0,1\}^*\} = L \\
\end{alignat*}

\textcolor{red}{Nach dem Bewertungsschema ist diese Lösung \textbf{8.5} Punkte wert.}

\subsubsection*{Automat, der 110100 akzeptiert}
\begin{figure}[H]
\centering
\begin{tikzpicture}

  \node[state,initial] (q_{lambda}) {$q_{\lambda}$};
  \node[state] (q_1) [right of=q_{lambda}] {$q_1$};
  \node[state] (q_{11}) [right of=q_1] {$q_{11}$};
  \node[state] (q_{110}) [right of=q_{11}] {$q_{110}$};
  \node[state,accepting] (q_{1100}) [right of=q_{110}] {$q_{1100}$};

  \path (q_{lambda}) edge[bend left, above]  node {$1$}   (q_1)
        (q_{lambda}) edge[loop above]        node {$0$}   (q_{lambda})
        (q_1)        edge[bend left, below]  node {$0$}   (q_{lambda})
        (q_1)        edge[bend left, above]  node {$1$}   (q_{11})
        (q_{11})     edge[bend left, above]  node {$0$}   (q_{110})
        (q_{11})     edge[loop above]        node {$1$}   (q_{11})
        (q_{110})    edge[bend left, above]  node {$0$}   (q_{1100})
        (q_{110})    edge[bend left, below,draw=red]  node {$\color{red}{1}$}   (q_{11})
        (q_{1100})   edge[loop above]        node {$0,1$} (q_{1100});
\end{tikzpicture}
\end{figure}

\begin{alignat*}{2}
&\text{Klasse}[q_{\lambda}] && = \{\lambda\} \cup \{x0 \mid x \in \{0,1\}^* \text{ und $x0$ enthält das Teilwort $1100$ nicht}\} \\
&\text{Klasse}[q_{1}] && = \{x1 \mid x \in \{0,1\}^* \text{ und $x1$ enthält das Teilwort $1100$ nicht}\}  \\
&\text{Klasse}[q_{11}] && = \{x11 \mid x \in \{0,1\}^* \text{ und $x11$ enthält das Teilwort $1100$ nicht}\} \\
&\text{Klasse}[q_{110}] && = \{x110 \mid x \in \{0,1\}^* \text{ und $x110$ enthält das Teilwort $1100$ nicht}\} \\
&\text{Klasse}[q_{1100}] && = \{x1100y \mid x,y \in \{0,1\}^*\} = L \\
\end{alignat*}

\textcolor{red}{Nach dem Bewertungsschema ist diese Lösung \textbf{8} Punkte wert.}

\end{document}