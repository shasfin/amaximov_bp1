% ------------------------------------------------------------------------
% file `arbeitsblatt-definition-12-exercise-body.tex'
%
%     exercise of type `definition' with id `12'
%
% generated by the `definition' environment of the
%   `xsim' package v0.11 (2018/02/12)
% from source `arbeitsblatt' on 2020/10/28 on line 283
% ------------------------------------------------------------------------
Auf Wörter definieren wir folgende Operationen:
\begin{description}
    \item[Länge:] Sei \(w\) ein Wort über \(\Sigma\). \(\abs{w}\) bezeichnet die Länge des Wortes. Zum Beispiel, \(\abs{\text{aabbcc}}=6\).
    \item[Anzahl Symbole:] Sei \(w\) ein Wort über \(\Sigma\). \(\abs{w}_a\) bezeichnet die Anzahl der Vorkommnisse des Symbols \(a\) im Wort \(w\). Zum Beispiel, \(\abs{\text{aaabbc}}_a=3\).
    \item[Konkatenation:] Seien \(v\) und \(w\) zwei Wörter über \(\Sigma\). \(vw\) bezeichnet das Wort, welches mit \(v\) anfängt und mit \(w\) endet. Man bezeichnet \(v\) als Präfix und \(w\) als Suffix von \(vw\). Zum Beispiel, wenn wir \texttt{001} mit \texttt{01} konkatenieren, erhalten wir \texttt{00101}.
    \item[Wiederholung:] Sei \(a \in \Sigma\) und \(n \in \mathbb{N}\). \(a^n\) bezeichnet das Wort, welches aus \(n\) Wiederholungen von \(a\) besteht. Zum Beispiel, \(\text{0}^4\) bezeichnet das Wort \texttt{0000}.
\end{description}
